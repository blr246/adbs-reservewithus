\documentclass[letterpaper]{article}%
\input{tex-util/misc.tex}
\input{tex-util/figures.tex}
\input{tex-util/homeworks.tex}
\input{tex-util/listings.tex}
\input{tex-util/math.tex}
\input{tex-util/pagesetup.tex}

\def\CourseTitle{Advanced Database Systems}
\def\CourseNumber{CSCI-GA.2434-001}
\def\Instructor{Prof. Dennis Shasha}
\def\HomeworkSetNumber{2}

\author{%
  Brandon Reiss \& Kevin Mullin
}

\date{Tuesday, November 19th, 2013}
\title{%
  \CourseTitle{} - Homework \HomeworkSetNumber{} Supplement \\
  {\large \Instructor{}} \\
  {\large \CourseNumber{}}
}

\lhead{\CourseNumber{} - Problem Set \HomeworkSetNumber{}}
\chead{}
\rhead{Reiss \& Mullin, \thepage{} of \pageref{LastPage}}
\fancyfoot{}

\begin{document}
\noindent
\maketitle
\thispagestyle{empty}

\begin{enumerate}[1.]
    \begin{problemcopy} \item The Undo, No Redo algorithm has the following
        description: Transactions First transfer before-images of each page to
        the audit trail and then put the after-images in the database.

        Suppose we changed it so that we transfer the after-image to the
        database before putting the before image to the audit train. Will this
        be correct? Prove your answer.
    \end{problemcopy}

    \begin{problemcopy} \item Suppose that a transaction manager has done the
        following steps in the two phase commit protocol for a transaction T:
        \begin{enumerate}[(a)]
          \item It has asked all servers whether they are willing to commit and
            they all responded affirmatively.
          \item It has told all servers to commit, but has not waited for a
            response.
        \end{enumerate}
        Does the transaction manager still need to keep a record of transaction
        T? If not, why not? If so, for how long?
    \end{problemcopy}

    \begin{problemcopy} \item Suppose we are trying to decide whether to put a
        non-clustering index on attribute B for relation R to support equality
        selections on B. The relation has 10 million records. Each page can
        store 10 records. There are 10,000 different values of B. A sequential
        scan will fetch 10 pages per read. Explain why you would or would not
        want to include a non-clustering index.
    \end{problemcopy}

    Given that R has 10,000,000 records and each read of 10 pages can fetch 100
    records at 10 records per page, a full scan of R takes 100,000 reads.

    There are two main possibilities affecting our use of a non-clustering
    index on R.

    \begin{enumerate}[i)]
      \item \textbf{The values of B are distributed approximately uniformly.}
        In the case when the distinct values of B are distributed evenly over
        the records of R, each distinct value of B will have approximately
        1,000 records. In this case, we expect a non-clustering index to
        outperform a sequential scan when there are fewer than 100 equality
        selections on distinct values of B since that will require access to
        fewer than 100,000 records.
        
        The reason that we look at records rather than pages in this case is
        because the most degenerate case is when each row of R matched by the
        equality selection on B lies on a distinct page, and so we are forced
        to perform a read for every row. The non-clustering index does not make
        any guarantees about co-locality of records on shared pages.

      \item \textbf{The values of B are distributed non-uniformly.} There are
        many types of distributions with a high degree of skew that could
        induce a very large number of disk reads for certain equality
        selections on B. The proper strategy depends on the nature of the
        distribution.
        
        The most skewed distribution possible is the one in which all but one
        distinct value of B matches 1 record and then the final distinct value
        of B matches 9,990,001 records. If the heavily loaded value of B is
        unlikely to appear in an equality selection, then this skewed
        distribution is great for a non-clustering index since it is impossible
        to exceed 100,000 reads if we avoid the value of B having 9,990,001
        records.

        Between the uniform case and the most skewed case above, there are many
        other distributions whose first-order moment is the required 1,000
        records but with varying second-order moments. An adversarial case when
        applying the non-clustering index is any distribution where larger
        batches of records on values of B are likely to be chosen given the
        particular query on B. To underperform the scan, the union of the sets
        of records gathered by the equality selections must exceed 100,000.
    \end{enumerate}

    \begin{problemcopy} \item Suppose that each of relations R, S, T, V, W has
        A as its only key. Which of the following queries may output a
        different number of records if DISTINCT is removed? Prove your answer.
    \end{problemcopy}

    \begin{enumerate}[a.]
        \begin{problemcopy}
        \item \texttt{SELECT DISTINCT R.A, S.A \\
            FROM R, A \\
            WHERE R.B = S.C}
        \end{problemcopy}

        DISTINCT is unnecessary since both R and S are privileged relations.

        \begin{problemcopy}
        \item \texttt{SELECT DISTINCT R.A \\
            FROM R, S \\
            WHERE R.B = S.C}
        \end{problemcopy}

        DISTINCT is necessary since only R is privileged and S does not reach
        R.

        \begin{problemcopy}
        \item \texttt{SELECT DISTINCT R.A \\
            FROM R, S
            WHERE R.B = S.A}
        \end{problemcopy}

        DISTINCT is unnecessary since R is privileged and S REACHES R.

        \begin{problemcopy}
        \item \texttt{SELECT DISTINCT R.A \\
            FROM R, S, T, V, W \\
            WHERE R.B = S.A \\
            AND R.D = T.D \\
            AND R.C = V.A \\
            AND T.A = S.B \\
            AND W.A = S.D}
        \end{problemcopy}

        DISTINCT is unnecessary since R is privileged, S REACHES R, V REACHES
        R, T REACHES S, W REACHES S. The REACHES relation is transitive, so all
        unprivileged tables S, T, V, W reach R.

        \begin{problemcopy}
        \item \texttt{SELECT DISTINCT R.A \\
            FROM R, S, T \\
            WHERE R.B = T.A \\
            AND R.B = S.C}
        \end{problemcopy}

        DISTINCT is necessary since R is privileged and T REACHES R. Relation S
        is unprivileged and does not reach R.

    \end{enumerate}

    \begin{problemcopy} \item Suppose that the following four transactions are
        the only ones that execute during some interval (R stands for read, W
        for write, and different letter arguments represent different data
        items).

        \texttt{%
          T1: R(A) W(B) R(C) W(D) \\
          T2: W(A) W(C) \\
          T3: R(D) R(B) \\
          T4: R(A) R(E)%
        }
    \end{problemcopy}

    \begin{enumerate}[a.]
        \begin{problemcopy}
        \item What is the finest chopping of T1 assuming that its reads and
          writes cannot be reordered? Show the chopping.
        \end{problemcopy}

        The finest chopping of T1 when its reads and writes cannot be reordered
        is

        \begin{center}
          \framebox{\texttt{R(A) W(B) R(C) W(D)}}\hspace{1mm}.
        \end{center}
        \vspace{1em}

        The reason that T1 cannot be chopped is that insertion of a single
        \textsl{sibling} edge will split at least one of the pairs of
        operations (\texttt{R(A), R(C)}) or (\texttt{W(B), W(D)}), leading to
        an SC cycle between at least one of
        \begin{equation*}
          \text{(\texttt{T11--T2--T12--T11}) or (\texttt{T11--T3--T12--T11}).}
        \end{equation*}

        \begin{problemcopy}
        \item What is the finest chopping of T1 assuming that its reads and
          writes can be reordered? Show the reordering, then the chopping.
        \end{problemcopy}

        The finest chopping of T1 when reads and writes can be reordered is

        \begin{center}
          T1$^*$ = \framebox{\texttt{R(A) R(C) W(B) W(D)}} \\
          \vspace{2mm}
          T11 = \framebox{\texttt{R(A) R(C)}}\hspace{1mm},\hspace{2mm}
          T12 = \framebox{\texttt{W(B) W(D)}}\hspace{1mm}.
        \end{center}
        \vspace{1em}

        The reason that T1 may be chopped no further is that any further
        chopping of T11 introduces an SC cycle
        \begin{equation*}
          \text{(\texttt{T111--T2--T112--T111})\phantom{.}}
        \end{equation*}
        and any further chopping of T12 introduces an SC cycle
        \begin{equation*}
          \text{(\texttt{T121--T3--T122--T121}).}
        \end{equation*}

        \begin{problemcopy}
        \item Can any other transactions be chopped?
        \end{problemcopy}

        T2 cannot be chopped because it introduces an SC cycle \texttt{(T1--T21--T22--T1)}.
        \vspace{0.5em}

        T3 cannot be chopped because it introduces an SC cycle \texttt{(T1--T31--T32--T1)}.
        \vspace{0.5em}

        T4 can be chopped into
        T41 = \framebox{\texttt{R(A)}}\hspace{1mm},\hspace{2mm}
        T42 = \framebox{\texttt{R(E)}}\hspace{2mm}\vspace{1mm}
        since T42 has no conflict with any other transaction, and so there is
        no way to complete an SC cycle.

    \end{enumerate}

\end{enumerate}

\end{document}

